\documentclass[]{article}
\usepackage{amsmath}

%opening
\title{ADIPCV Assignment}
\author{}

\begin{document}

\maketitle

For instructions regarding running code, kindly refer to the \texttt{README.md} file attached.

\section{Computing Homographies}
In the first part, the keypoints can be selected
manually using the GUI. Homography matrices are computed
with respect to the first image (\textit{Nov 17}), and
applied. The computation of the homography is done by
using Direct Linear Transform (DLT), solving the equation
$\mathbf{A}\mathbf{H} = \mathbf{0}$. $\mathbf{H}$ is computed
by calculating the null space of $\mathbf{A}$, the
eigenvector corresponding to the smallest eigenvalue.

Applying the homography matrix is done by using $\mathbf{X} = \mathbf{H} \mathbf{x}$ in the homogeneous coordinate space,
and taking into consideration the boundary conditions.

\section{Using SIFT}
The keypoint selection is conducted using SIFT. In order to
remove outliers, \textit{Lowe's ratio test} is used to only use the keypoints that are closer to each other. Post this,
the homography computation and application follow the same scheme as the earlier section.

\section{Scene Summarization}
Each of the images are projected with respect to the first image,
and differences are calculated between the applied images. These
differences can be used to assess the growth in every consecutive image.
Since pixel wise differences may not reflect true changes due to factors such as \textbf{illumination}, it could be prudent 
to alternatively apply a mean shifting, histogram equalization, or utilize the gradients instead to visualize the scene changes.

\section{Vanishing Points \& Affine Rectification}
The vanishing line is computed using points selected by the user
in a GUI driven program. Points are selected on parallel lines in the same plane, basically in the following manner:
\begin{enumerate}
  \item Select 8 points, 2 each on 2 sets of parallel lines
  \item The points of intersection of these lines is computed 
  \item Following this, the vanishing line can be computed.
\end{enumerate}
These intersection computations are basically a series of cross
products. 

Post this, the vanishing line $<l_1, l_2, l_3>$ can be used to 
find the affine rectification matrix as:
\begin{equation}
  \mathbf{H} = 
  \begin{bmatrix}
  1 & 0 & 0 \\
  0 & 1 & 0 \\
  l_1 & l_2 & l_3 \\
  \end{bmatrix}
\end{equation}
\end{document}
