\documentclass[]{article}
\usepackage{amsmath}

%opening
\title{ADIPCV Assignment}
\author{}

\begin{document}

\maketitle

For instructions regarding running code, kindly refer to the \texttt{README.md} file attached.

\section{Computing Fundamental Matrix}
The fundamental matrix is computed using the \textbf{8 point algorithm}. It consists of
computing keypoints, normalizing them, taking a DLT and denormalizing the null space to retrieve our result.
In order to reduce the effect of noise into our computations,
we consider 10 points for the DLT. The points are normalized using
\begin{equation}
  \mathbf{x_{norm}} = \frac{\mathbf{x}-\mu}{\sigma}
\end{equation}

Using Direct Linear Transform (DLT),
$\mathbf{A}\mathbf{F} = \mathbf{0}$. $\mathbf{F}$ is computed
by calculating the null space of $\mathbf{A}$, the
eigenvector corresponding to the smallest eigenvalue.

Post this, the SVD of $\mathbf{F}$ is calculated, and the smallest eigenvalue is forced
to zero, and then $\mathbf{F}$ is recomputed.

Also, the fundamental matrix is accordingly denormalized, using:
\begin{equation}
  \mathbf{T} = 
  \begin{bmatrix}
    \frac{1}{\sigma_x} & 0 & \frac{-\mu_x}{\sigma_x} \\
    0 & \frac{1}{\sigma_y} & \frac{-\mu_y}{\sigma_y} \\
  0 & 0 & 1 \\
  \end{bmatrix}
\end{equation}
Similarly, for $\mathbf{T_p}$. Thus,
\begin{equation}
  \mathbf{F_{denorm}} = {T_p}^{T} \mathbf{F} T 
\end{equation}

\section{Drawing Epipolar Lines}
For each of the keypoints, using the fundamental matrix,
the corresponding epipolar lines can be determined, using $F$ and $F^T$ respectively.
Using boundness of the image plane, we can plot the lines.

\section{Epipoles}
The epipoles can be computed from the right and left null spaces of the fundamental matrix,
and from the intersection of the epipolar lines. The two of them are quite close to each other,
thus verifying our computations.

\section{Projective Matrices}
The projective matrices assumed in this case are:
\begin{equation}
  \mathbf{P_1} = [ \mathbf{I} | \mathbf{0} ]
\end{equation}
and
\begin{equation}
  \mathbf{P_2} = [ \mathbf{[e_2]_X} \mathbf{F} | \mathbf{e_2} ]
\end{equation}

\section{Depth and Scene Point Computation}
The scene points can be computed using DLT, resulting from the equation
$\mathbf{X} \times \mathbf{P} x$. Also, the depth can be computed from
the equation:
\begin{equation}
  d = \left( \mathbf{X} - \mathbf{C_i} \right) . {mr^3}_i
\end{equation}
where $\mathbf{X}$ is the scene point, $\mathbf{C_i}$ is the centre of the camera,
and ${mr^3}_i$ is the 3rd row of the $\mathbf{M}$, the submatrix of the camera matrix.
\end{document}
